%!TEX root = ../paper.tex
\section{Conclusion}
\label{sec:conclusion}

In this project, we tried to reproduce the latest results in the video classification research area.
The first challenge was in the data preprocessing and preprepation task.
As regenerating the data takes several hours, these decisions are important and must be made with care.
We received our best results when extracting with 15 frames per second, using 10-times stacked optical flows, using random frame cropping, and using 16 frames per video for the fusion part.
Our final prediction accuracy was \todo{Accuracy}.
However, we cannot rule out, that generating with more frames per second can yield better performance.
Also some of these decisions, such as the frame stack size and the number of frames per video, were based on resource limitations.

For the architecture, we based our nets on the well known Caffenet and VGG nets, which are well established in the literature.
The VGG net with \todo{layers, size etc.} ..